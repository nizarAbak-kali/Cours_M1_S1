LAB 3 Couche appli Messagerie DNS SNMP

1 Messagerie 

1.1 
1. BOB-> serveur messagerie -> Alice  (-> = envoie par protocole smtp +http +un autre pour les pieces jointes)


1.2 Service de messagerie 
1.2.1 Config  des serveur 

Commandes observees: http://www.cs.cf.ac.uk/Dave/PERL/node175.html


HELO
    -- Initiates a conversation with the mail server. When using this command you can specify your domain name so that the mail server knows who you are. For example, HELO mailhost2. cf.ac.uk. 
MAIL
    -- Indicates who is sending the mail. For example,

    MAIL FROM: <dave@cs.cf.ac.uk>.

    Remember this is not going to be your name -- it's the name of the person who is sending the mail message. Any returned mail will be sent back to this address.

RCPT
    -- Indicates who is recieving the mail. For example,

    RCPT TO: <user@email.com>. You can indicate more than one user by issuing multiple RCPT commands.

DATA
    -- Indicates that you are about to send the text (or body) of the message. The message text must end with the following five letter sequence: "\r\n.\r\n."

QUIT
    -- Indicates that the conversation is over.

EXPN
    -- Indicates that you are using a mailing list. 
HELP
    -- Asks for help from the mail server. 
NOOP
    -- Does nothing other than get a reponse from the mail server. RSETAborts the current conversation. 
SEND
    -- Sends a message to a user's terminal instead of a mailbox. 
SAML
    -- Sends a message to a user's terminal and to a user's mailbox. 
SOML
    -- Sends a message to a user's terminal if they are logged on; otherwise, sends the message to the user's mailbox. 
TURN
    -- Reverses the role of client and server. This might be useful if the client program can also act as a server and needs to receive mail from the remote computer. 
VRFY
    -- Verifies the existence and user name of a given mail address. This command is not implemented in all mail servers. And it can be blocked by firewalls. 

Every command will receive a reply from the mail server in the form of a three digit number followed by some text describing the reply. For example,

250 OK

500 Syntax error, command unrecognized.

The complete list of reply codes is shown below: (you'll never see most of them if you program your mail server correctly!!)

211
    -- A system status or help reply. 
214
    -- Help Message. 
220
    -- The server is ready. 
221
    -- The server is ending the conversation. 
250
    -- The requested action was completed. 
251
    -- The specified user is not local, but the server will forward the mail message. 
354
    -- This is a reply to the DATA command. After getting this, start sending the body of the mail message, ending with "\r\n.\r\n." 

CR: retour chariot

2)IMF : sujet , from , to ,content type , date , messae id , version min, X-mailer , content tranfert encoding , message text
3)c'est tres bien .

1.4.1
1. 

ISSUE:

POP commands to view email via a Telnet session. Should a message become stuck, it can be deleted manually by the user or administrator.

SOLUTION:

USER - 1st login command

PASS - 2nd login command

In session Commands

STAT - returns total number of messages and total size

LIST [message] - lists all messages or the specified message

DELE - deletes the specified message

TOP - returns the headers and number of lines from the message

RETR - retrieves the whole message

QUIT - logs out and saves any changes

KEYWORDS: WM3, TELNET, POP, COMMAND


2.


